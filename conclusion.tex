%!TEX root = main.tex


This paper revisited the solution-counting strategies for the well-known global cardinality constraint. It first highlights that in the computation of the initial result provided by~\cite{PesantQZ12}, a solution can be counted several times and, thus, is not an upper bound. We also provide a correct calculation of such an upper bound. Finally, we build a benchmark for the $gcc$ constraint with a original method, which had not been done in the previous work, and we compare heuristics based on both quantities (the previous calculation by \cite{PesantQZ12} and our upper bound) on this benchmark. We first show that the two estimators lead to different choices within counting-based strategies. Finally, we solve a number of instances with both heuristics and compare the results. In practice, the heuristics based on \cite{PesantQZ12} calculation performs slightly better on our benchmark: although it is not a upper bound, it is proved useful as an \textit{estimator} of the solution density.

 Though the former result is not an upper bound, it remains an estimator that can be used to guide the search. That is probably why the error was difficult to spot. It also reinforces the idea that the correlation between one estimator and the actual number of solutions is much more important that its accuracy. A future work on $\texttt{global\_cardinality}$ could be to determines other estimators based, for example, on probabilistic approaches.
 
 %The main issue of those estimators on $\texttt{global\_cardinality}$ constraints, is that they require a lot of complex operations. Every instantiation for each constraint is tested and propagated. Computing them each time a decision has to be made is very time consuming. Indeed, we need to compute the Lower Bound Graph, the Upper Bound Graph then we need to estimate the number of matchings on these graphs. This is why, in the presented experiments, the instantiations ordering is only made in the beginning of the search. Thus, at each branch of the search tree, we do not necessarily visit first the space where there are more solutions at this point of the search. That is why estimators that are faster to compute are needed for this constraint. 
 
 